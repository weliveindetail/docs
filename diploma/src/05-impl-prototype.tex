
Ziel der Implementierung des Prototypen ist es, einen Machbarkeitsnachweis f�r die in Kapitel \ref{sec:concept} erarbeiteten Grundfunktionalit�ten zu erbringen. Daf�r wurde die Realisierung des in \ref{sec:concept-sameprocess-stepping} beschriebenen Verfahrens zur Ausf�hrung des Pr�flings in Einzelschritten gew�hlt. Die Anbindung der Benutzeroberfl�che gestaltet sich damit einfach. Die Ausf�hrung eines Source-Level-Steps kann vom Benutzer durch das Bet�tigen einer Funktionstaste der Tastatur erfolgen. Die Implementierung kann sich damit auf die Umsetzung der Kernfunktionalit�ten konzentrieren. Da die Einzelschrittausf�hrung auf der Verwendung von Breakpoints basiert, demonstriert sie die Funktionsweise aller grundlegenden Mechanismen zur Ablaufkontrolle eines Trace-Debuggers. Die Einzelschrittausf�hrung soll zun�chst nur f�r den Audio-Handler (\ref{sec:context-core-execution}) des Pr�flings verf�gbar sein. Im Audio-Handler findet die Signalverarbeitung in Echtzeit statt. Auch im Hinblick auf die Problematik der Echtzeitverarbeitung werden damit alle Schwerpunkte abgedeckt.

Mit einem als \emph{Wire-Debugging} bezeichneten Mechanismus steht in Reaktor Core zudem bereits eine einfache M�glichkeit zur Inspektion des Pr�flings zur Verf�gung. Er soll zu den zu erstellenden Ablaufkontrollmechanismen kompatibel sein. Die Implementierung des Prototypen soll zuk�nftige Erg�nzungen erm�glichen. Dazu z�hlen explizite Breakpoints auf beliebigen Modulen, Step-Out und Step-Over Funktionalit�ten f�r Macros, das Debugging mehrerer Core Cells zugleich sowie die Modifikation von Laufzeitwerten. 

%Unter Verwendung des erarbeiteten Konzepts nicht realisierbar sind asynchrone Unterbrechungen der Ausf�hrung. 

%Anregungen zur spezifischen Erweiterung dieses Inspektionsmechanismus' finden sich im Ausblick in Kapitel \ref{sec:conclusion}. 

%\begin{itemize}

%  \item nur einfaches Stepping im AudioHandler
  
%  \item Planung soll k�nftige Erg�nzungen ber�cksichtigen, z.B.:
%  \begin{itemize}
%    \item Debugging beliebiger Handler (speziell des Init-Handlers)
%    \item Breakpoints auf beliebigen Modulen oder Ports (Inputs/Outputs)
%    \item Resume nach Breakpoint oder Einzelschritt
%    \item Step-Out und Step-Over f�r Macros (Voraussetzung: Debug-Modus f�r gesamte Core Cell, damit bei �nderung der Ansicht keine Neukompilierung n�tig ist)
%    \item Core-Cell-�bergreifendes Debugging
%    \item �nderung von Laufzeitwerten (Werte f�r Wire-Debugging m�ssten nicht nur als Kopie herausgeschrieben sondern auch zur Weiterverwendung wieder eingelesen werden)
%  \end{itemize}
  
%  \item Im aktuellen Konzept nicht vorgesehen/m�glich sind:
%  \begin{itemize}
%    \item asynchrones Anhalten der Ausf�hrung (wie "`Pause"' in Visual Studio), da Debug-Trap des Prozessors nicht verf�gbar
%    \item omniscient Debugging (nicht Future-Proof)
%  \end{itemize}
  
%  \item Vorerst ausgeklammert.
%	\begin{itemize}
%  	\item Inspektion mehrerer / verschiedener Core Cells gleichzeitig
%  	\item Behandlung von Haltepunkten im GUI-Thread
%	\end{itemize}

%\end{itemize}