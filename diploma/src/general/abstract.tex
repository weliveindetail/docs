%\begin{abstract}
\thispagestyle{empty}

%\addcontentsline{toc}{nonmainmatterchapter}{Zusammenfassung}
\vspace*{\fill}

\section*{Zusammenfassung}
\label{sec:abstract}

\noindent
Todo

%Zur Erzeugung und Ver�nderung von Kl�ngen kommen in digitalen Musikinstrumenten Signal- und Ereignisverarbeitungsverfahren zum Einsatz, die als wiederverwendbare Komponenten .. digitaler %Musikinstrumnte k�nnen mit Hilfe einer grafischen dom�nenspezifischen Programmiersprache .. Verarbeitungsabl�ufe .. Voraussetzung f�r Einsatz auf gew�hnlichen Personal Computern: Performanz

%Mit Hilfe eines Compiler werden diese Komponenten in Maschinencode �bersetzt, der zur Laufzeit direkt von der CPU ausgef�hrt wird. F�r Dom�nenexperten ist es damit m�glich Komponenten  ohne %Kenntnisse allgemeiner Programmiersprachen verwendet werden und erreicht dabei eine Performanz, die mit in C++ entwickelten  vergleichbar ist.

%. Neben der Evaluierung der Integrierbarkeit in das bestehende Projekt standen vor allem die speziellen Anforderungen an das Debugging der Echtzeitverarbeitung im Vordergrund.

%Die Reaktor Core Technologie ist Bestandteil der modularen Musikproduktions-Software Reaktor der Firma Native Instruments. Reaktor Core bezeichnet eine grafische dom�nenspezifische Programmiersprache zur Modellierung von Signal- und Ereignisverarbeitungsprozessen, die auf die Konstruktion von Komponenten digitaler Musikinstrumente zugeschnitten ist. Mit dem Reaktor Core Compiler werden diese Komponenten in Maschinencode �bersetzt, der zur Laufzeit direkt von der CPU ausgef�hrt wird. Reaktor Core kann von Dom�nenexperten auch ohne Kenntnisse allgemeiner Programmiersprachen verwendet werden und erreicht dabei eine Performanz, die mit in C++ entwickelten Komponenten vergleichbar ist.

%Zur Verbesserung von Inspektion und Fehlersuche wurde im Zuge der vorliegenden Diplomarbeit ein Prototyp eines Trace-Debuggers f�r den Reaktor Core Compiler entwickelt. Neben der Evaluierung der Integrierbarkeit in das bestehende Projekt standen vor allem die speziellen Anforderungen an das Debugging der Echtzeitverarbeitung im Vordergrund.

\vspace{\fill}
%\end{abstract}