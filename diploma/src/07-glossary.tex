&\\

Echtzeit-Thread& 
Vom Scheduler des Betriebssystems speziell priorisierter Thread
\newline \\

Enkopplungspunkt& 
Funktion in der Ausf�hrungsabfolge des Echtzeit-Threads, an der die Umleitung des Aufrufstacks stattfindet
\newline \\
  
Implementieren eines Breakpoints& 
�berschreiben eines Befehls im Maschinencode des Pr�flings durch den Befehlscode des Breakpoints. Der urspr�ngliche Befehlscode wird zuvor zwischengespeichert.
\newline \\

Instruction-Level&
Programmcode auf Maschinenebene
\newline \\

Lineares Segment& 
Sequenzielle Folge von Maschinenbefehlen ohne Sprunganweisungen. Nur der erste Maschinenbefehl eines linearen Segments kann ein Sprungziel sein.
\newline \\

Programmdaten&
Informationen zum Pr�flingsprogramm, die eine Inspektion des Pr�flings auf Quellcodeebene erm�glichen. Sie werden i.\,A. vom Compiler gesammelt und nach Abschluss des �bersetzungsvorgangs dem Debugger zur Verf�gung gestellt.
\newline \\

Regul�rer Stackspeicherbereich& 
Speicherbereich des Aufrufstacks eines Threads im Stack Segment des Prozesses
\newline \\

Separater Stackspeicher& 
Speziell allozierter Speicherbereich f�r die Aufnahme der Stackframes aller nach der Umleitung des Aufrufstacks aufgerufenen Funktionen (im Heap Segment des Prozesses)
\newline \\

Sequenzieller Befehlsblock& 
Sequenzielle Folge von Maschinenbefehlen, die an einem Sprungziel beginnt und alle Maschinenbefehle bis zum n�chsten Sprungziel enth�lt. Gegen�ber den Grundbl�cken im Compilerbau k�nnen sequenzielle Befehlsbl�cke Sprunganweisungen an beliebigen Stellen enthalten.
\newline \\

Setzen eines Breakpoints& 
Registrierung eines Breakpoints gegen�ber dem Debugger
\newline \\

Source-Level&
Programmcode auf dem Abstraktionsniveau einer h�heren Programmiersprache
\newline \\

Umleitung des Aufrufstacks& 
Manipulaion der Stackregister des Prozessors. Die Stackframes aller nachfolgend aufgerufenen Funktionen werden in einem separaten Stackspeicher aufgebaut\newline \\

Zeitschranke& 
Fest vorgegebener Zeitpunkt an dem die Ergebnisse einer Berechnung vorliegen m�ssen und die Ausf�hrung im Echtzeit-Thread zur�ckgekehrt sein muss
\newline \\

%Begriff& 
%Beschreibung
%\newline \\
